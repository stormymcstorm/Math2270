\documentclass{../mathhomework}
\usepackage{enumitem}
\usepackage{graphicx}
\usepackage{float}

\newcommand{\Let}[1]{\textit{let }#1}
\newcommand{\Powerset}[1]{\mathcal{P}(#1)}
\newcommand{\By}[1]{\text{(by #1)}}
\newcommand{\circnum}[1]{\text{\textcircled{#1}}}

\newenvironment{Mat}{\begin{bmatrix}}{\end{bmatrix}}

% Assigment Info
\coursetitle{Linear Algebra}
\courseinstructor{Professor MacArthur}

\student{Carson Storm}

\assignmenttitle{Homework \#2}
\assignmentduedate{September 4, 2019}

\begin{document}
\maketitle

\pagebreak

% Section 1.4 #1, 3, 5, 7, 11, 15, 19, 21, 25
% Section 1.5 #1, 3, 7, 11, 13, 15, 19, 25, 27, 39

\begin{problem}[1.4\#1]
    Compute the product using (a) the definition, as in Example 1, and (b) the row–vector rule for computing $A\Vect{x}$. If a product is undefined, explain why.

    \begin{equation*}
        \begin{Mat}
            -4 & 2 \\
            1 & 6 \\
            0 & 1
        \end{Mat}
        \begin{Mat}
            3 \\ -2 \\ 7
        \end{Mat}
    \end{equation*}

    \begin{solution}
        The product is undefined because the number of columns in $A$ is not equal to the number of rows in $\Vect{x}$.
    \end{solution}
\end{problem}

\begin{problem}[1.4\#3]
    Compute the product using (a) the definition, as in Example 1, and (b) the row–vector rule for computing $A\Vect{x}$. If aproduct is undefined, explain why.

    \begin{equation*}
        \begin{Mat}
            6 & 5 \\
            -4 & 4 \\
            7 & 6
        \end{Mat}
        \begin{Mat}
            2 \\ -3
        \end{Mat}
    \end{equation*}

    \begin{solution}[Part A:]
        \begin{equation*}
            \begin{Mat}
                6 & 5 \\
                -4 & 4 \\
                7 & 6
            \end{Mat}
            \begin{Mat}
                -2 \\ 3
            \end{Mat} =
            2 \begin{Mat} 6 \\ -4 \\ 7 \end{Mat}
            - 3 \begin{Mat} 5 \\ -3 \\ 6 \end{Mat}
            =
            \begin{Mat}
                -3 \\ 1 \\ -4 
            \end{Mat}
        \end{equation*}
    \end{solution}

    \begin{solution}[Part B:]
        \begin{equation*}
            \begin{Mat}
                2(6) - 3(5) \\
                2(-4) - 3(-3) \\
                2(7) - 3(6) \\
            \end{Mat} = \begin{Mat}
                -3 \\ 1 \\ -4
            \end{Mat}
        \end{equation*}
    \end{solution}
\end{problem}

\begin{problem}[1.4\#5]
    Use the definition of $A\Vect{x}$ to write the matrix equation as a vector equation.

    \begin{equation*}
        \begin{Mat}
            5 & 1 & -8 & 4 \\
            -2 & -7 & 3 & -5
        \end{Mat} \begin{Mat}
            5 \\ -1 \\ 3 \\ -2
        \end{Mat} = \begin{Mat}
            -8 \\ 16
        \end{Mat}
    \end{equation*}

    \begin{solution}
        \begin{equation*}
            5\begin{Mat}5\\-2\end{Mat} - \begin{Mat}1\\-7\end{Mat} + 3\begin{Mat}-8\\3\end{Mat} - 2\begin{Mat}4\\-5\end{Mat} = \begin{Mat}-8\\16\end{Mat}
        \end{equation*}
    \end{solution}
\end{problem}

\begin{problem}[1.4\#7]
    Use the definition of $A\Vect{x}$ to write the vector equation as a matrix equation.

    \begin{equation*}
        \begin{Mat}
            4 \\ -1 \\ 7 \\ -4
        \end{Mat}x_1 + \begin{Mat}
            -5 \\ 3 \\ -5 \\ 1
        \end{Mat}x_2 + \begin{Mat}
            7 \\ -8 \\ 0 \\ 2
        \end{Mat}x_3 = \begin{Mat}
            6 \\ -8 \\ 0 \\ -7
        \end{Mat}
    \end{equation*}

    \begin{solution}
        \begin{equation*}
            \begin{Mat}
                4 & -5 & 7 \\
                -1 & 3 & -8 \\
                7 & -5 & 0 \\
                -4 & 1 & 2
            \end{Mat}
            \begin{Mat}
                x_1 \\ x_2 \\ x_3
            \end{Mat}
            =
            \begin{Mat}
                6 \\ -8 \\ 0 \\ -7
            \end{Mat}
        \end{equation*}
    \end{solution}
\end{problem}

\begin{problem}[1.4\#11]
    Given $A$ and $\Vect{b}$, write the augmented matrix for the linear system that corresponds to the matrix equation $A\Vect{x} = \Vect{b}$. Then solve the system and write the solution as a vector.

    \begin{equation*}
        A = \begin{Mat}
            1 & 2 & 4 \\
            0 & 1 & 5 \\
            -2 & -4 & -3
        \end{Mat}, \Vect{b} = \begin{Mat}
            -2 \\ 2 \\ 9
        \end{Mat}
    \end{equation*}

    \begin{solution}
        \begin{align*}
            \begin{Mat}
            1 & 2 & 4 & -2 \\
            0 & 1 & 5 & 2 \\
            -2 & -4 & -3 & 9
            \end{Mat} & \equiv
            \begin{Mat}
            1 & 2 & 4 & -2 \\
            0 & 1 & 5 & 2 \\
            0 & 0 & 5 & 5  
            \end{Mat} & R3 = 2*R1 + R3 \\ & \equiv
            \begin{Mat}
            1 & 2 & 4 & -2 \\
            0 & 1 & 0 & -3 \\
            0 & 0 & 5 & 5  
            \end{Mat} & R2 = -R3 + R2 \\ & \equiv
            \begin{Mat}
            1 & 2 & 4 & -2 \\
            0 & 1 & 0 & -3 \\
            0 & 0 & 1 & 1  
            \end{Mat} & R3 = \frac{1}{5}R3 \\ & \equiv
            \begin{Mat}
            1 & 0 & 4 & 4 \\
            0 & 1 & 0 & -3 \\
            0 & 0 & 1 & 1  
            \end{Mat} & R1 = -2 * R2 + R1 \\ & \equiv
            \begin{Mat}
            1 & 0 & 0 & 0 \\
            0 & 1 & 0 & -3 \\
            0 & 0 & 1 & 1  
            \end{Mat} & R1 = -4 * R3 + R1 
        \end{align*}

        \begin{equation*}
            \Vect{x} = \begin{Mat}
                0 \\ -3 \\ 1
            \end{Mat}
        \end{equation*}
    \end{solution}
\end{problem}

\begin{problem}[1.4\#15]
    Let $A = \begin{Mat}
        2 & -1 \\
        -6 & 3
    \end{Mat}$ and $\Vect{b} = \begin{Mat}
        b_1 \\ b_2
    \end{Mat}$. Show that the equation $A\Vect{x} = \Vect{b}$ does not have a solution for all possible $\Vect{b}$, 
    and describe the solution set of all $\Vect{b}$ for which $A\Vect{x} = \Vect{b}$ \textit{does} has a solution.

    \begin{solution}
        \begin{align*}
            \begin{Mat}
                2 & -1 & b_1 \\
                -6 & 3 & b_2
            \end{Mat} & \equiv
            \begin{Mat}
                2 & -1 & b_1 \\
                0 & 0 & 3b_1 + b_2
            \end{Mat} & R2 = 3 * R1 + R2
        \end{align*}

        $A\Vect{x} = \Vect{b}$ has a solution for all $b_1$ and $b_2$ such that $3b_1 + b_2 = 0$.
    \end{solution}
\end{problem}

\begin{problem}[1.4\#19]
    Can each vector in $\R^4$ be written as a linear combination of the columns of the matrix $A$? Do the columns of $A$ span $\R^4$?
    \begin{equation*}
        A = \begin{Mat}
            1 & 3 & 0 & 3 \\
            -1 & -1 & -1 & 1 \\
            0 & -4 & 2 & -8 \\
            2 & 0 & 3 & -1
        \end{Mat}
    \end{equation*}

    \begin{solution}
        \begin{align*}
            \begin{Mat}
                1 & 3 & 0 & 3 \\
                -1 & -1 & -1 & 1 \\
                0 & -4 & 2 & -8 \\
                2 & 0 & 3 & -1
            \end{Mat} & \equiv
            \begin{Mat}
                1 & 3 & 0 & 3 \\
                0 & 2 & -1 & 4 \\
                0 & -4 & 2 & -8 \\
                2 & 0 & 3 & -1
            \end{Mat} & R2 = R1 + R2 \\ & \equiv
            \begin{Mat}
                1 & 3 & 0 & 3 \\
                0 & 2 & -1 & 4 \\
                0 & 0 & 0 & 0 \\
                2 & 0 & 3 & -1
            \end{Mat} & R3 = 2 * R2 + R3
        \end{align*}

        No, the columns of $A$ do not span $\R^4$ because they are not linearly independent.
    \end{solution}
\end{problem}

\pagebreak
\begin{problem}[1.4\#21]
    Let $\Vect{v_1} = \begin{Mat}
        1 \\ 0 \\ -1 \\ 0
    \end{Mat}$, $\Vect{v_2} = \begin{Mat}
        0 \\ -1 \\ 0 \\ 1
    \end{Mat}$, $\Vect{v_3} = \begin{Mat}
        1 \\ 0 \\ 0 \\ -1
    \end{Mat}$. Does $\Set{\Vect{v_1}, \Vect{v_2}, \Vect{v_3}}$ span $\R^3$? Why or why not?

    \begin{solution}
        \begin{align*}
            \begin{Mat}
                1 & 0 & 1 \\
                0 & -1 & 0 \\
                -1 & 0 & 0 \\
                0 & 1 & -1
            \end{Mat} & \equiv
            \begin{Mat}
                1 & 0 & 1 \\
                0 & -1 & 0 \\
                0 & 0 & 1 \\
                0 & 1 & -1
            \end{Mat} & R3 = R1 + R3 \\ & \equiv
            \begin{Mat}
                1 & 0 & 1 \\
                0 & 0 & -1 \\
                0 & 0 & 1 \\
                0 & 1 & -1
            \end{Mat} & R2 = R4 + R2 \\ & \equiv
            \begin{Mat}
                1 & 0 & 1 \\
                0 & 0 & -1 \\
                0 & 0 & 0 \\
                0 & 1 & -1
            \end{Mat} & R3 = R2 + R3 \\ & \equiv
            \begin{Mat}
                1 & 0 & 1 \\
                0 & 1 & -1 \\
                0 & 0 & 0 
                0 & 0 & -1 \\
            \end{Mat} & \text{Swap R2 \& R4}\\ & \equiv
            \begin{Mat}
                1 & 0 & 1 \\
                0 & 1 & -1 \\
                0 & 0 & -1 \\
                0 & 0 & 0 
            \end{Mat} & \text{Swap R3 \& R4}\\ & \equiv
            \begin{Mat}
                1 & 0 & 1 \\
                0 & 1 & -1 \\
                0 & 0 & 1 \\
                0 & 0 & 0 
            \end{Mat} & R3 = -R3 \\ & \equiv
            \begin{Mat}
                1 & 0 & 1 \\
                0 & 1 & 0 \\
                0 & 0 & 1 \\
                0 & 0 & 0 
            \end{Mat} & R2 = R3 + R2 \\ & \equiv
            \begin{Mat}
                1 & 0 & 0 \\
                0 & 1 & 0 \\
                0 & 0 & 1 \\
                0 & 0 & 0 
            \end{Mat} & R1 = R1 - R2
        \end{align*}

        No, $\Set{\Vect{v_1}, \Vect{v_2}, \Vect{v_3}}$ does not spans all of $\R^3$ because there is not a pivot position in each row.
    \end{solution}
\end{problem}

\begin{problem}[1.4\#25]
    Note that $\begin{Mat}
        4 & -3 & 1 \\
        5 & -2 & 5 \\
        -6 & 2 & -3
    \end{Mat}\begin{Mat}
        -3 \\ -1 \\ 2
    \end{Mat} = \begin{Mat}
        -7 \\ -3 \\ 10
    \end{Mat}$. Use this fact (and no row operations) to find scalars $c_1, c_2, c_3$ such that
    $\begin{Mat}-7\\-3\\10\end{Mat} = c_1\begin{Mat}4\\5\\-6\end{Mat} + c_2\begin{Mat}-3\\-2\\2\end{Mat} + c_3\begin{Mat}1\\5\\-3\end{Mat}$.

    \begin{solution}
        The scalars are $c_1 = -3, c_2 = -1, c_3 = 2$.
    \end{solution}
\end{problem}

\begin{problem}[1.5\#1]
    Determine if the system  has a nontrival solution. Try to use as few row operations as possible.

    \begin{align*}
        2x_1 - 5x_2 + 8x_3 &= 0 \\
        -2x_2 - 7x_2 + x_3 &= 0 \\
        4x_1 + 2x_2 + 7x_3 &= 0
    \end{align*}

    \begin{solution}
        \begin{align*}
            \begin{Bmatrix}
                2x_1 - 5x_2 + 8x_3 = 0 \\
                -2x_2 - 7x_2 + x_3 = 0 \\
                4x_1 + 2x_2 + 7x_3 = 0
            \end{Bmatrix} & \equiv
            \begin{Mat}
                2 & -5 & 8 & 0 \\
                -2 & -7 & 1 & 0 \\
                4 & 2 & 7 & 0
            \end{Mat} \\ & \equiv
            \begin{Mat}
                2 & -5 & 8 & 0 \\
                0 & -12 & 9 & 0 \\
                4 & 2 & 7 & 0
            \end{Mat} & R2 = R1 + R2 \\ & \equiv
            \begin{Mat}
                2 & -5 & 8 & 0 \\
                0 & -12 & 9 & 0 \\
                0 & 12 & -9 & 0
            \end{Mat} & R3 = -2 * R1 + R3 \\ & \equiv
            \begin{Mat}
                2 & -5 & 8 & 0 \\
                0 & -12 & 9 & 0 \\
                0 & 0 & 0 & 0
            \end{Mat} & R3 = R2 + R3
        \end{align*}

        The system has a nontrival solution because there is a free variable.
    \end{solution}
\end{problem}

\begin{problem}[1.5\#3]
    Determine if the system  has a nontrival solution. Try to use as few row operations as possible.

    \begin{align*}
        -3x_1 + 5x_2 - 7x_3 &= 0 \\
        -6x_1 + 7x_2 + x_3 &= 0
    \end{align*}

    \begin{solution}
        The system has a nontrival solution because it is undetermined, so it has a free variable.
    \end{solution}
\end{problem}

\begin{problem}[1.5\#7]
    Describe all solutions of $A\Vect{x} = \Vect{0}$ in parametric form, where $A$ is row equivalent to the given matrix.

    \begin{equation*}
        \begin{Mat}
            1 & 3 & -3 & 7 \\
            0 & 1 & -4 & 5
        \end{Mat}
    \end{equation*}

    \begin{solution}
        \begin{align*}
            A = \begin{Mat}
                1 & 3 & -3 & 7 \\
                0 & 1 & -4 & 5
            \end{Mat} & =
            \begin{Mat}
                1 & 0 & 9 & -8 \\
                0 & 1 & -4 & 5
            \end{Mat} & R1 = -3 * R2 + R1 \\
            A\Vect{x} = 0 & \equiv
            \begin{Bmatrix}
                x_1 + 9x_3 - 8x_4 = 0 \\
                x_2 - 4x_3 + 5x_4 = 0
            \end{Bmatrix} \\ & \equiv
            \begin{Bmatrix}
                x_1 = 8x_4 - 9x_3 \\
                x_2 = 4x_3 - 5x_4
            \end{Bmatrix} \\
            & \implies \Vect{x} = x_3\begin{Mat}-9\\4\\1\\0\end{Mat} + x_4\begin{Mat}8\\-5\\0\\1\end{Mat}
        \end{align*}
    \end{solution}
\end{problem}

\pagebreak
\begin{problem}[1.5\#11]
    Describe all solutions of $A\Vect{x} = \Vect{0}$ in parametric form, where $A$ is row equivalent to the given matrix.

    \begin{equation*}
        \begin{Mat}
            1 & -4 & -2 & 0 & 3 & -5 \\
            0 & 0 & 1 & 0 & 0 & -1 \\
            0 & 0 & 0 & 0 & 1 & -4 \\
            0 & 0 & 0 & 0 & 0 & 0
        \end{Mat}
    \end{equation*}

    \begin{solution}
        \begin{align*}
            A = \begin{Mat}
                1 & -4 & -2 & 0 & 3 & -5 \\
                0 & 0 & 1 & 0 & 0 & -1 \\
                0 & 0 & 0 & 0 & 1 & -4 \\
                0 & 0 & 0 & 0 & 0 & 0
            \end{Mat} & =
            \begin{Mat}
                1 & -4 & 0 & 0 & 3 & -7 \\
                0 & 0 & 1 & 0 & 0 & -1 \\
                0 & 0 & 0 & 0 & 1 & -4 \\
                0 & 0 & 0 & 0 & 0 & 0
            \end{Mat} & R1 = 2 * R2 + R1 \\ & =
            \begin{Mat}
                1 & -4 & 0 & 0 & 0 & 5 \\
                0 & 0 & 1 & 0 & 0 & -1 \\
                0 & 0 & 0 & 0 & 1 & -4 \\
                0 & 0 & 0 & 0 & 0 & 0
            \end{Mat} & R1 = -3 * R23+ R1 \\
            A\Vect{x} = 0 & \equiv 
            \begin{Bmatrix}
                x_1 - 4x_2 + 5x_6 = 0 \\
                x_3 - x_6 = 0 \\
                x_5 - 4x_6 = 0
            \end{Bmatrix} \\ & \equiv
            \begin{Bmatrix}
                x_1 = 4x_2 - 5x_6 \\
                x_3 = x_6 \\
                x_5 = 4x_6
            \end{Bmatrix} \\
            & \implies \Vect{x} = x_2\begin{Mat}4\\1\\0\\0\\0\\0\end{Mat} + x_4\begin{Mat}0\\0\\0\\1\\0\\0\end{Mat}  + x_6\begin{Mat}-5\\0\\1\\0\\4\\1\end{Mat}
        \end{align*}
    \end{solution}
\end{problem}

\begin{problem}[1.5\#13]
    Suppose the solution set of a certain linear system of equations can be described as $x_1 = 5 + 4x_3$, $x_2 = -2 -7x_3$, with $x_3$ free. Use vectors to describe this set as a line in $\R^3$.

    \begin{solution}
        \begin{equation*}
            \Vect{x}(t) = t\begin{Mat}4\\-7\\1\end{Mat} + \begin{Mat}5\\-2\\0\end{Mat}
        \end{equation*}
    \end{solution}
\end{problem}

\pagebreak
\begin{problem}[1.5\#15]
    Follow the method of Example 3 to describe the solution  of the following system in parametric form. 
    Also, give a geometric description of the solution set and compare it to that in Exercise 5.

    \begin{align*}
        x_1 + 3x_2 + x_3 & = 1 \\
        -4x_1 - 9x_2 + 2x_3 & = -1 \\
        -3x_2 - 6x_3 &= -3
    \end{align*}

    \begin{solution}
        \begin{align*}
            \begin{Bmatrix}
                x_1 + 3x_2 + x_3  = 1 \\
                -4x_1 - 9x_2 + 2x_3  = -1 \\
                -3x_2 - 6x_3 = -3
            \end{Bmatrix} & \equiv
            \begin{Mat}
                1 & 3 & 1 & 1 \\
                -4 & -9 & 2 & -1 \\
                0 & -3 & -6 & -3
            \end{Mat} \\ & \equiv
            \begin{Mat}
                1 & 3 & 1 & 1 \\
                0 & 3 & 6 & 3 \\
                0 & -3 & -6 & -3
            \end{Mat} R2 = 4*R1 + R2 \\ & \equiv
            \begin{Mat}
                1 & 3 & 1 & 1 \\
                0 & 3 & 6 & 3 \\
                0 & 0 & 0 & 0
            \end{Mat} R3 = R2 + R3 \\ & \equiv
            \begin{Bmatrix}
                x_1 - 5x_3 = 2 \\
                x_2 + 2x_3 = 1
            \end{Bmatrix} \\ & \equiv
            \begin{Bmatrix}
                x_1 = 5x_3 -2 \\
                x_2 = 1 - 2x_3
            \end{Bmatrix} \\ & \equiv
            \Vect{x}(t) = t\begin{Mat}5\\-2\\1\end{Mat} + \begin{Mat}-2\\1\\0\end{Mat}
        \end{align*}

        The solution set to this system is a line in $\R^3$ that goes through$\begin{Mat}
            -2\\1\\0
        \end{Mat}$ and parallel to $\begin{Mat}
            5\\-2\\1
        \end{Mat}$.
    \end{solution}
\end{problem} 

\begin{problem}[1.5\#19]
    Find the parametric equation of the line through $\Vect{a}$ and parallel to $\Vect{b}$.

    \begin{equation*}
        \Vect{a} = \begin{Mat}
            -2 \\ 0
        \end{Mat}, \Vect{b} = \begin{Mat}
            -5 \\ 3
        \end{Mat}
    \end{equation*}

    \begin{solution}
        \begin{equation*}
            \Vect{x}(t) = t\Vect{b} + \Vect{a} = t\begin{Mat}-5\\3\end{Mat}  + \begin{Mat}-2\\0\end{Mat}
        \end{equation*}
    \end{solution}
\end{problem}

\pagebreak
\begin{problem}[1.5\#25]
    Prove the second part of Theorem 6: Let $\Vect{w}$ be any solution of $A\Vect{x} = \Vect{b}$, and define $\Vect{v_h} = \Vect{w} - \Vect{p}$. Show that $\Vect{v_h}$ is a solution of $A\Vect{x} = \Vect{0}$.
    This shows that every solution of $A\Vect{x} = \Vect{b}$ has the form $\Vect{w} = \Vect{p} + \Vect{v_h}$, with $\Vect{p}$ a particular solution of $A\Vect{x} = \Vect{b}$ and $\Vect{v_h}$ a solution of $A\Vect{x} = 0$.

    \begin{solution}
        \begin{proof}
            Given $\Vect{p}$ is a solution to $A\Vect{x} = \Vect{b}$, and $\Vect{w}$ is a solution to $A\Vect{x} = \Vect{b}$, $\Vect{v_h} = \Vect{w} - \Vect{p}$ is a solution to $A\Vect{x} = 0$

            \begin{align}
                0 & = A\Vect{v_h} \\
                & = A(\Vect{w} - \Vect{p}) &\By{substitution}\\
                & = A\Vect{w} - A\Vect{p} &\By{distributive property}\\
                0 & = \Vect{b} - A\Vect{p} &\By{substitution}\\
                A\Vect{p} & = \Vect{b} &\By{addition}\\
                \Vect{b} & = \Vect{b} &\By{substitution}
            \end{align}
        \end{proof}

        Therefore if $\Vect{p}$ is a solution to $A\Vect{x} = \Vect{b}$, and $\Vect{w}$ is a solution to $A\Vect{x} = \Vect{b}$, then $\Vect{v_h} = \Vect{w} - \Vect{p}$ is a solution to $A\Vect{x} = 0$
    \end{solution}
\end{problem}

\begin{problem}[1.5\#27]
    Suppose $A$ is a $3 \times 3$ \textit{zero} matrix (with all zero entries). Describe the solution set of the equation $A\Vect{x} = \Vect{0}$.

    \begin{solution}
        The solution set of the equation $A\Vect{x} = 0$ when $A = \begin{Mat}
            0 & 0 & 0\\
            0 & 0 & 0\\
            0 & 0 & 0
        \end{Mat}$ is all vectors in $\R^3$.
    \end{solution}
\end{problem}

\begin{problem}[1.5\#39s]
    Let $A$ be an $m \times n$ matrix, and let $\Vect{u}$ be a vector in $\R^n$ that satisfies the equation $A\Vect{x} = \Vect{0}$. Show that for any scalar $c$ the vector $c\Vect{u}$ also satisfies $A\Vect{x} = \Vect{0}$. [That is, show that $A(c\Vect{u}) = \Vect{0}$].

    \begin{solution}
        Given $A\Vect{u} = 0$

        \begin{align*}
            A(c\Vect{u}) & = (A\Vect{u})c & \By{Associative property} \\
            & = (0)c & \By{substitution} \\
            & = 0 & \By{multiplication of zero}
        \end{align*}
    \end{solution}
\end{problem}

\end{document}
