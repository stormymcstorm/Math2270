\documentclass{../mathhomework}
\usepackage{enumitem}
\usepackage{graphicx}
\usepackage{float}

\newcommand{\Span}[1]{\textit{Span}\{#1\}}
\newcommand{\Vect}[1]{\pmb{#1}}
\newenvironment{Mat}{\begin{bmatrix}}{\end{bmatrix}}

% Assigment Info
\coursetitle{Linear Algebra}
\courseinstructor{Professor MacArthur}

\student{Carson Storm}

\assignmenttitle{Homework \#3}
\assignmentduedate{September 11, 2019}

% Section 1.9 #1, 3, 4, 5, 10, 11, 15, 17, 20, 21, 25
% Section 2.1 #1, 2, 3, 5, 7, 9, 13, 15, 21, 27
% Section 2.2 #1, 5, 7, 9, 13, 15, 17, 19, 21, 31

\begin{document}
\maketitle

\pagebreak

%%%% SECTION 1.9 %%%%

\begin{problem}[1.9\#1]
    Assuming $T$ is a linear transformation. Find the standard matrix of $T$ where $T:\R^2 \to \R^4, T(\Vect{e_1}) = (3,1,3,1)$ and $T(\Vect{e_2}) = (-5,2,0,0)$.
\end{problem}

\begin{problem}[1.9\#3]
    Assuming $T$ is a linear transformation. Find the standard matrix of $T$ where $T:\R^2 \to \R^2$ rotates poing (about the origin) through $3\pi / 2$ radians (counterclockwise).
\end{problem}

\begin{problem}[1.9\#5]
    Assuming $T$ is a linear transformation. Find the standard matrix of $T$ where $T:\R^2 \to \R^2$ is a vertical shear transformation that maps $\Vect{e_1}$ into $\Vect{e_1} - 2\Vect{e_2}$ but leaves the vector $\Vect{e_2}$ unchanged.
\end{problem}

\begin{problem}[1.9\#10]
    Assuming $T$ is a linear transformation. Find the standard matrix of $T$ where $T:\R^2 \to \R^2$ first reflects points through the vertical $x_2$-axis and then rotates points $\pi / 2$ radians.
\end{problem}

\begin{problem}[1.9\#11]
    Assuming $T$ is a linear transformation. Find the standard matrix of $T$ where $T:\R^2 \to \R^2$ first reflects points through the $x_1$-axis and then reflects points through the $x_2$-axis.
    Show that $T$ can also be describd as a linear transformation that roates points about the origin. What is the angle of that rotation?
\end{problem}

\begin{problem}[1.9\#15]
    Fill in the missing entries of the matrix, assuming that the equation holds for all values of the variables.

    \begin{equation*}
        \begin{Mat}
            ? & ? & ? \\
            ? & ? & ? \\ 
            ? & ? & ?
        \end{Mat}
        \begin{Mat}
            x_1 \\ x_2 \\ x_3
        \end{Mat} =
        \begin{Bmatrix}
            3x_1 - 2x_3 \\ 4x_1 \\ x_1 - x_2 + x_3
        \end{Bmatrix}
    \end{equation*}
\end{problem}

\begin{problem}[1.9\#17]
    Show that $T$ is a linear transformation by finiding a matrix that implments the mapping. Note that $x_1,x_2,\ldots$ are not vectors but are entries in vectors.

    \begin{equation*}
        T(x_1,x_2,x_3,x_4) = (0, x_1 + x_2, x_2 + x_3, x_3 + x_4)
    \end{equation*}
\end{problem}

\begin{problem}[1.9\#20]
    Show that $T$ is a linear transformation by finiding a matrix that implments the mapping. Note that $x_1,x_2,\ldots$ are not vectors but are entries in vectors.

    \begin{equation*}
        T(x_1,x_2,x_3,x_4) = 2x_1 + 3x_3 - 4x_4
    \end{equation*}
\end{problem}

\begin{problem}[1.9\#21]
    Show that $T$ is a linear transformation by finiding a matrix that implments the mapping. Note that $x_1,x_2,\ldots$ are not vectors but are entries in vectors.
    Let $T: \R^2 \to \R^5$ such that $T(x_1, x_2) = (x_1 + x_2, 4x_1 + 5x_2)$. Find $\Vect{x}$ such that $T(\Vect{x}) = (3,8)$.
\end{problem}

\begin{problem}[1.9\#25]
    Determine if the linear transformation in Exercise 17 is (a) one-to-one and (b) onto. Justfiy your answer.
\end{problem}


%%%% SECTION 2.1 %%%%

\begin{problem}[2.1\#1]
    Compute each matrix sum or product if it defined. If an expression is undefined, explain why.
    \begin{equation*}
        A = \begin{Mat}
            2 & 0 & -1 \\ 
            4 & -5 & 2
        \end{Mat}, B = \begin{Mat}
            7 & -5 & 1 \\ 
            1 & -4 & -3
        \end{Mat}, C = \begin{Mat}
            1 & 2 \\ 
            -2 & 1
        \end{Mat}, D = \begin{Mat}
            3 & 5 \\ 
            -1 & 4
        \end{Mat}, E = \begin{Mat}
            -5 \\ 3
        \end{Mat}
    \end{equation*}

    Find $-2A, B - 2A, AC$ and $CD$.
\end{problem}

\begin{problem}[2.1\#2]
    Compute each matrix sum or product if it defined. If an expression is undefined, explain why.
    \begin{equation*}
        A = \begin{Mat}
            2 & 0 & -1 \\ 
            4 & -5 & 2
        \end{Mat}, B = \begin{Mat}
            7 & -5 & 1 \\ 
            1 & -4 & -3
        \end{Mat}, C = \begin{Mat}
            1 & 2 \\ 
            -2 & 1
        \end{Mat}, D = \begin{Mat}
            3 & 5 \\ 
            -1 & 4
        \end{Mat}, E = \begin{Mat}
            -5 \\ 3
        \end{Mat}
    \end{equation*}

    Find $-A + 2B, 3C - E, CB$ and $EB$.
\end{problem}

\begin{problem}[2.1\#3]
    Let $A = \begin{Mat}
        4 & -1 \\ 5 & -2
    \end{Mat}$. Compute $3I_2 - A$ and $(3I_2)A$.
\end{problem}

\begin{problem}[2.1\#5]
    Compute the product $AB$ in two ways: (a) by the definition, where $A\Vect{b_1}$ and $A\Vect{b_2}$ are computed separately, and (b) by the row-column rule for computing $AB$.
\end{problem}

\begin{problem}[2.1\#7]
    If a matrix $A$ is $5 \times 3$ and the product $AB$ is $5 \times 7$, what is the size of $B$.
\end{problem}

\begin{problem}[2.1\#9]\end{problem}

\begin{problem}[2.1\#13]\end{problem}

\begin{problem}[2.1\#15]\end{problem}

\begin{problem}[2.1\#21]\end{problem}

\begin{problem}[2.1\#27]\end{problem}


%%%% SECTION 2.2 %%%%

\begin{problem}[2.2\#1]\end{problem}

\begin{problem}[2.2\#5]\end{problem}

\begin{problem}[2.2\#7]\end{problem}

\begin{problem}[2.2\#9]\end{problem}

\begin{problem}[2.2\#13]\end{problem}

\begin{problem}[2.2\#15]\end{problem}

\begin{problem}[2.2\#17]\end{problem}

\begin{problem}[2.2\#19]\end{problem}

\begin{problem}[2.2\#21]\end{problem}

\begin{problem}[2.2\#31]\end{problem}


\end{document}